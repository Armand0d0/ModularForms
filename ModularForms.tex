\documentclass[letterpaper,10pt]{article}

% Language setting
% Replace `english' with e.g. `spanish' to change the document language
\usepackage[utf8]{inputenc}
\usepackage[T1]{fontenc}
\usepackage[english]{babel}

% Set page size and margins
% Replace `letterpaper' with `a4paper' for UK/EU standard size
\usepackage[letterpaper,top=2cm,bottom=2cm,left=3cm,right=3cm,marginparwidth=1.75cm]{geometry}

% Useful packages
\usepackage{amsmath}
\usepackage{amssymb}
\usepackage{tikz-cd}
%\usepackage{stmaryrd}
\usepackage{graphicx}
\graphicspath{ {./} }

\usepackage{biblatex} %Imports biblatex package
%\usepackage[colorlinks=true, allcolors=blue]{hyperref}

\addbibresource{bibliography.bib} %Import the bibliography file


\title{Modular Forms, Elliptic curves and Modular curves}
\author{Armand Perrin, Samy, Damian}


\begin{document}

\maketitle%---------------------------------------------------------------

\newcommand{\C}{\ensuremath{\mathbb{C}}}
\newcommand{\Z}{\ensuremath{\mathbb{Z}}}

\newcommand{\SL}{\ensuremath{SL_2(\mathbb{Z})}}

\part{The Modularity Theorem}

\paragraph{}The modularity theorem was first partially proven by Andrew Wiles in 1995 (before known as the Taniyama-Shimura conjecture) 
and led to a proof of Fermat's Last theorem (1637). The proof was then completed in 2001, showing a strong connexion between 
two kind of objects : modular forms and elliptic curves. In this part we will introduce some notions in order to understand 
the statement of the modularity theorem : 

\paragraph{Theorem } (Modularity Theorem) All elliptic curves over $\mathbb{Q}$ are modular.



\section{Riemann surfaces}%------------------------------------------------------------------------------------------------------------------------------------

\paragraph{Definition 1.1} A $n$-manifold $\mathcal{M}$ is a Hausdorff topological space locally homeomorphic to $\mathbb{R}^n$.
This means that for every point $p \in \mathcal{M}$ there exists an open neighborhood $U$ of $p$ and an homeomorphism $\varphi : U \to \mathbb{R}^n$ to an open subset
of $\mathbb{R}^n$. $(U,\varphi)$ is called a chart.


\paragraph{Definition 1.2} Given two charts $(U,\varphi)$ and $(V,\psi)$ of an $n$-manifold $\mathcal{M}$ with $U\cap V \neq \emptyset$ the map
$\varphi \circ \psi^{-1} : \psi(U\cap V) \subset \mathbb{R}^n \to \varphi(U\cap V) \subset \mathbb{R}^n$ is called a transition map.

{\itshape \paragraph{} We want to extend theses notions to the complex numbers therfore we can replace $\mathbb{R}^n$ by $\mathbb{C}^n$ in the
 two previous definitions to define a complex $n$-manifold. A real $n$-manifold is said to be $\mathcal{C}^k$ if it's transition maps are all $\mathcal{C}^k$ 
 similarly we will consider the complex 1-manifolds whose transition maps are not only smooth but also holomorphic, wich is a much more stronger constraint.}

\paragraph{Definition 1.3} A Riemann surface is a connected complex 1-manifold whose transitions maps are holomorphic.

\paragraph{Definition 1.4} A map $f : \mathcal{M}_1 \to \mathcal{M}_2$ between two Riemann surfaces is said to be holomorphic if for each charts $(U,\varphi),(V,\psi)$
of $\mathcal{M}_1,\mathcal{M}_2$ respectively, the complex function $\psi \circ f \circ \varphi^{-1}$ is holomorphic over $\varphi(U\cap f^{-1}(V))$.

\paragraph{Definition 1.5} (Unformal) \\ Intuitively the genus of a Riemann surface is the number of its holes as a multiple torus. 

\includegraphics[scale=0.7]{genus}

{\itshape \paragraph{} More formally the genus of $X$ can be defined as half the dimension of $H_1(X,\mathbb{C})$, the first singular homology group.  } 


\paragraph{Theorem 1.6} (Classification of compact Riemann surfaces) \\ 
Two compact Riemann surfaces are homeomorphic to eachother if and only if they have the same genus.


\section{Lattices and complex tori}%------------------------------------------------------------------------------------------------------------------------------------

\paragraph{Definition 2.1} A \textbf{lattice} is a subset of $\mathbb{C}$ of the form $\omega_1\mathbb{Z} \oplus \omega_2\mathbb{Z}$ with $\{\omega_1,\omega_2\}$ a 
$\mathbb{R}$-basis of $\mathbb{C}$. 
{\itshape \paragraph{} This is equivalent to  $\frac{\omega_1}{\omega_2} \notin \mathbb{R}$. We can suppose that $\frac{\omega_1}{\omega_2} \in \mathcal{H}$ for convenience.
}
\paragraph{Definition 2.2} We say that two lattices $\Lambda$ and $\Lambda'$ are isomorphic if there exists an $m\in \mathbb{C}$ such that 
$m\Lambda = \Lambda'$.

\paragraph{Proposition 2.3} Every lattice $\Lambda = \omega_1\mathbb{Z} \oplus \omega_2\mathbb{Z}$ is isomorphic to  $\mathbb{Z} \oplus \tau\mathbb{Z}$ 
for some $\tau \in \mathcal{H}$.\\

{\itshape Proof :} $\omega_1 \neq 0$ so we can consider the isomorphic lattice $(\frac{1}{\omega_1})\Lambda = \mathbb{Z} \oplus \tau\mathbb{Z}$ with 
$\tau = \frac{\omega_2}{\omega_1}$, if $\tau \notin \mathcal{H}$ then $-\tau \in \mathcal{H}$ and $\mathbb{Z} \oplus -\tau\mathbb{Z} = \mathbb{Z} \oplus \tau\mathbb{Z}$.

\paragraph{Proposition 2.4} Two lattices $\Lambda = \mathbb{Z} \oplus \tau\mathbb{Z}$ and $\Lambda' = \mathbb{Z} \oplus \tau'\mathbb{Z}$ are isomorphic if and only if
$\tau' = \gamma \cdot \tau$ for some $\gamma \in SL_2(\mathbb{Z})$.




\paragraph{Definition 2.5} The complex tori associated to a lattice $\Lambda$ is the abelian quotient group $\mathbb{C} / \Lambda$.  

\paragraph{Proposition 2.6} The quotient topology over $\mathbb{C} / \Lambda$ makes it a Riemann surface.

\paragraph{Definition 2.7} An \textbf{isomorphism} of complex tori $\varphi : \mathbb{C} / \Lambda \to \mathbb{C} / \Lambda'$ is a holomorphic 
group isomorphism. $\mathbb{C} / \Lambda$ and $\mathbb{C} / \Lambda'$ are then said to be isomorphic.

\paragraph{Proposition 2.8} Any complex tori isomorphism $\varphi : \mathbb{C} / \Lambda \to \mathbb{C} / \Lambda'$ is of the form : $\varphi(z) = mz$ 
with $m\Lambda = \Lambda'$.
{\itshape \paragraph{}Proof: } Here we use a lifting theorem : there exists a holomorphic map $\widetilde{\varphi}: \C \to \C$ that lifts $\varphi$. Meaning that 
$\varphi(z + \Lambda) = \widetilde{\varphi}(z) + \Lambda'$. See \cite{munkres} 
for details. For any $\lambda \in \Lambda$ the map $z\mapsto \widetilde{\varphi}(z + \lambda) - \widetilde{\varphi}(z)$ is continuous but takes it's values in $\Lambda'$ 
that is discrete therefore it is constant. Thus, $\widetilde{\varphi}'(z + \lambda) = \widetilde{\varphi}'(z)$, so $\widetilde{\varphi}'$ is $\Lambda$-periodic. But by 
Liouville's theorem a bounded holomorphic function is constant. Integrating gives that $\widetilde{\varphi}$ is a degre on polynomial, i.e there exists $m,a,\in \C$ such 
that $\widetilde{\varphi}(z) = mz + a$. But since $\varphi$ is also a group isomorphism, we have $0 + \Lambda' = \varphi(0 + \Lambda) = \widetilde{\varphi}(0) + \Lambda' 
 = a + \Lambda'$. So $a\in \Lambda'$. Similarly, for $\lambda \in \Lambda \; \widetilde{\varphi}(\lambda) = \varphi(\Lambda) = \Lambda'$. So $m\Lambda \subset \Lambda'$. 
 If the inclusion was strict there would be some $z\in \Lambda'$ such that $z/m \notin \Lambda$ but $\varphi(z/m + \Lambda) = \Lambda'$ and Ker$\varphi  = \Lambda$ by 
 injectivity. So $m\Lambda = \Lambda'$.
 Reducing modulo $\Lambda'$ we get $\varphi(z + \Lambda) = mz + \Lambda'= m(z + \Lambda)$.


\section{Complex tori and elliptic curves}%------------------------------------------------------------------------------------------------------------------------------------
\paragraph{} In this section we note $\Lambda = \omega_1\mathbb{Z} \oplus \omega_2\mathbb{Z}$ a fixed lattice.

\paragraph{Definition 3.1} Given a sub-field $\mathbb{K}$ of $\mathbb{C}$, an \textbf{elliptic curve} over $\mathbb{K}$ is the set of points $(x,y) \in \mathbb{K}^2$ such that 
\[\ y^2 = 4x^3 - ax - b \] for some $(a,b) \in \mathbb{K}^2$ such that $a^3-27b^2 \neq 0$.
{\itshape \paragraph{} Remark: The condition $a^3-27b^2 \neq 0$ correspond to the curve being smooth.
}

\paragraph{Definition 3.2} An isogeny of elliptic curves is a map  between two elliptic curves such that $ \phi(x,y) = (R_1(x,y),R_2(x,y))$ 
with $R_1$ and $R_2$ two rational fuctions and that is also a group morphism. And an isogeny that is also a group isomorphism is called an isomorphism of
elliptic curves.

\paragraph{Proposition 3.3} If there is an isogeny $\phi$ from $E_1$ to $E_2$ then there is an isogeny
$\psi : E_1 \to E_2$ called the dual isogeny, and we say that $E_1$ and $E_2$ are isogenuous.
This defines an equivalence relation.

{\itshape \paragraph{}Remark: It can be shown that every curve that is the zero set of a polynomial of the form : 
\[ y^2z - (ax^3+bx^2z+ cxz^2 + dz^3)\]
is isogenuous (if we adapt the previous definition) to a curve of the form \[ y^2z - (x^3+bxz^2 + cz^3)\]
whose solutions of the form (x,y,1) satisfies the equation : \[\ y^2 = x^3 + bx + c \]

}



\paragraph{Definition 3.4} The Weierstrass function $\wp$ is defined by : 
\[\ \wp(z) = \frac{1}{z^2} +  \sum_{\omega \in \Lambda , \omega \neq 0 }\frac{1}{(z - \omega)^2} - \frac{1}{\omega^2} \]
\paragraph{Proposition 3.5} \textbf{ }\\ $(i)\;\;\; \wp$ is even \\  $(ii)\;\;\wp$ and $\wp'$ are $\Lambda$-periodic.

{\itshape \paragraph{} Proof :}
For $(i)$ the sum is even because : \[\sum_{\omega \in \Lambda , \omega \neq 0 }\frac{1}{(-z - \omega)^2} - \frac{1}{\omega^2} = 
 \sum_{\omega \in \Lambda , \omega \neq 0 }\frac{1}{(z -(- \omega))^2} - \frac{1}{\omega^2}\] and $\omega \mapsto -\omega$ is 
 a permutation of $\Lambda\backslash \{0\}$\\.
For $(ii)$, from the definition we derive : \[ \wp'(z) = -2\sum_{\omega\in\Lambda} \frac{1}{(z-\omega)^3} \]
And for $i\in \{1,2\}$, we have : \begin{align*} \wp'(z + \omega_i) &= -2\sum_{\omega\in\Lambda} \frac{1}{(z + \omega_i-\omega)^3}\\
  &= -2\sum_{\omega\in\Lambda} \frac{1}{(z - (\omega-\omega_i))^3}\\
  &= -2\sum_{\omega\in\Lambda - \omega_i} \frac{1}{(z - \omega)^3}\\
  &= \wp'(z)\\
\end{align*}%\begin{align*}\end{align*}
Since $\omega \mapsto \omega-\omega_i$ is a permutation of $\Lambda$. Thus $\wp'$ is  $\Lambda$-periodic.
It follows that for $z\mapsto \wp(z + \omega_i) - \wp(z)$ has 0 derviative and is therefore constant. But it's value at $-\omega_i/2$ is 
$\wp(\omega_i/2) - \wp(-\omega_i/2) = 0$ by parity.

\paragraph{Theorem 3.6} The Weierstrass function satisfies the differential equation : 

\[\ \wp'^2 = 4\wp^3 - g_2\wp - g_3 \] for two complex numbers $g_2$ and $g_3$ that only depend on $\Lambda$.

{\itshape \paragraph{} Proof :} For $z$ close to 0 we have : \begin{align*}
    \wp(z) &= \frac{1}{z^2} +  \sum_{\omega \in \Lambda , \omega \neq 0 }\frac{1}{\omega^2}  \Bigl(\frac{1}{(1 - \frac{z}{\omega})^2} - 1\Bigr) \\
     &=  \frac{1}{z^2} +  \sum_{\omega \in \Lambda , \omega \neq 0 }\frac{1}{\omega^2}  \sum_{n = 1}^\infty (n+1)\frac{z^n}{\omega^n}\\
     &\text{We can change summation order as the series converges absolutely}\\
     &=  \frac{1}{z^2} +  \sum_{n = 1}^\infty (n+1)z^n  \sum_{\omega \in \Lambda , \omega \neq 0 } \frac{1}{\omega^2}\frac{1}{\omega^n} \\
     &=  \frac{1}{z^2} +  \sum_{n = 1}^\infty (n+1)z^n G_{n+2}  \\
     &=  \frac{1}{z^2} +  \sum_{n = 1}^\infty (2n+1) G_{2n+2} z^{2n} \;\; \text{  since  } \forall n \geq 1 \; G_{2n+1} = 0  \\
\end{align*}
Therefore we can compute : \begin{align*}
  \wp'(z) &= -\frac{2}{z^3} + \sum_{n = 1}^\infty (2n+1)2n G_{2n+2} z^{2n-1} \\
  \wp'(z)^2 &= \frac{4}{z^6} - \frac{4}{z^3}\sum_{n = 1}^\infty (2n+1)2n G_{2n+2} z^{2n-1} + \Bigl(\sum_{n = 1}^\infty (2n+1)2n G_{2n+2} z^{2n-1}\Bigr)^2\\
  &= \frac{4}{z^6} - \frac{24G_4}{z^2} + h(z) \;\;\;\; \text{ with h holomorphic on a neighborhood of 0}
\end{align*} On the other hand :
\begin{align*}
  \wp(z) &= \frac{1}{z^2} + 3G_4z^2 + O(z^4) \\
  \wp(z)^3 &= \frac{1}{z^6} + \frac{9G_4z^2}{z^4} + O(1) \\
   4\wp(z)^3 &= \frac{4}{z^6} + \frac{36G_4}{z^2} + g(z)  \;\;\;\; \text{ with g holomorphic on a neighborhood of 0}\\
\end{align*} Then we obtain that $\wp' - 4\wp^3 + g_2\wp$ is holomorphic on a neighborhood of 0 for $g_2 := 60G_4$. But by $\Lambda$-periodicity this function is holomorphic on 
 $\mathbb{C}$. By continuity and $\Lambda$-periodicity it is also bounded on $\mathbb{C}$ and therefore constant.
 Let's denote -$g_3$ this constant (one can show with similar calculations that $g_3 = 140G_6$). We have then :
  \[ \wp'^2 = 4\wp^3 - g_2\wp - g_3  \]

\paragraph{Lemma 3.7} If $f$ is a $\Lambda$-periodic meromorphic function, then :   \\$(i)$ $f$ has as many zeros as poles (counting multiplicity). \\
$(ii)$ $f$ takes each value of $\C$ the same number of time on $\C / \Lambda$.\\
$(iii)$ $\sum_{x\in \C /\Lambda } x\; v_x(f) = 0$ in $\C/\Lambda$ where $v_x(f )$ is the order of $f$ at $x$.\\
$(iv)$ The Weierstrass function takes each value of $\mathbb{C}$ exactly twice at $z + \Lambda$ and $-z + \Lambda$.
{\itshape \paragraph{} Proof :} Let $ D:=\{a\omega_1+b\omega_2 \in \mathbb{C} \;|\; a,b\in [0,1]\}$ and $\partial D$ it's counterclockwise border. 
Since the function $g: z\mapsto f'(z)/f(z)$ 
is meromorphic on $D$ wich is compact, it has finitely many poles on $D$. Therefore there exists $a\in\C$ such that g has no poles on $\gamma := a + \partial D$.
Let's for example assume that $\gamma(0) = 0, \gamma(1/4) = \omega_1, \gamma(1/2) = \omega_1 + \omega_2, \gamma(3/4) = \omega_2.$
This way we have by $\Lambda$-periodicity of $g$ that $\gamma(t)-\omega_1 = \gamma(1 + 1/4 -t) \;\;\forall t\in [1/4,1/2]$ and 
$\gamma(t)-\omega_2 = \gamma(3/4 -t)\;\; \forall t\in [1/2,3/4]$.\\
Let's show that $\frac{1}{2i\pi}\int_{a + \partial D} g(z)dz = 0$ : 

\begin{align*} \int_{a + \partial D} g(z)dz &= \int_{0}^1 g(\gamma(t))\gamma'(t)dt \\
  &= \int_{0}^{1/4} g(\gamma(t)) \omega_1 dt +  \int_{1/4}^{1/2} g(\gamma(t)) \omega_2 dt +
    \int_{1/2}^{3/4} g(\gamma(t)) (-\omega_1) dt +  \int_{3/4}^{1} g(\gamma(t)) (-\omega_2) dt \\
  &= \int_{0}^{1/4} g(\gamma(t)) \omega_1 dt +  \int_{1/4}^{1/2} g(\gamma(t)- \omega_1) \omega_2 dt +
    \int_{1/2}^{3/4} g(\gamma(t) - \omega_2) (-\omega_1) dt +  \int_{3/4}^{1} g(\gamma(t)) (-\omega_2) dt \\
  &= \int_{0}^{1/4} g(\gamma(t)) \omega_1 dt +  \int_{1/4}^{1/2} g(\gamma(1+\frac{1}{4} -t)) \omega_2 dt +
    \int_{1/2}^{3/4} g(\gamma(\frac{3}{4} - t)) (-\omega_1) dt +  \int_{3/4}^{1} g(\gamma(t)) (-\omega_2) dt \\
  &= \int_{0}^{1/4} g(\gamma(t)) \omega_1 dt +  \int_{3/4}^{1} g(\gamma(t)) \omega_2 dt +
    \int_{0}^{1/4} g(\gamma(t)) (-\omega_1) dt +  \int_{3/4}^{1} g(\gamma(t)) (-\omega_2) dt \\
  &= 0
\end{align*}
Then by the Argument Principle, $f$ has as many zeros as poles (with multiplicity). If $k$ denotes the number of poles of $f$, then for any $c\in \C$, $f-c$ is also a 
$\Lambda$-periodic meromorphic function with $k$ poles, therefore it has $k$ zeros and $f$ takes $k$ times the value $c$. 
In particular for any $c \in \C$, $\wp$ takes the value $c$ exactly twice 
because it's only pole is 0 and of multiplicity 2. And by parity it has to be in points of the form  $z + \Lambda$ and $-z + \Lambda$.\\
To prove $(iii)$ we compute with similar method that : 
\[ \int_{a + \partial D} zg(z)dz = \int_{0}^{1/4} \gamma(t)g(\gamma(t)) \omega_1 dt +  \int_{3/4}^{1} \gamma(\frac{5}{4} -t)g(\gamma(t)) \omega_2 dt +
  \int_{0}^{1/4} \gamma(\frac{3}{4} - t)g(\gamma(t)) (-\omega_1) dt +  \int_{3/4}^{1}\gamma(t) g(\gamma(t)) (-\omega_2) dt\]

And since $\forall t\in [0,1/4] \; \gamma(\frac{3}{4} - t) - \gamma(t)  = \omega_2$ and $\forall t\in [3/4,1] \; \gamma(t) - \gamma(\frac{5}{4} - t)  = \omega_1$ we have : 

\begin{align*} \int_{a + \partial D} zg(z)dz &= \omega_2\int_{0}^{1/4} g(\gamma(t)) \omega_1 dt + \omega_1\int_{3/4}^{1} g(\gamma(t)) (-\omega_2) dt \\
    &= \omega_2\int_{[0,\omega_1]} g(z) dz - \omega_1\int_{[0,\omega_2]} g(z)  dz\\
\end{align*}
By studying the function $h : t \mapsto exp(\int_{[0,t]}g(z)dz)$ we can show that   $h(0) = h(\omega_1) = h(\omega_2)= 1$
Therefore $\int_{[0,\omega_i]} g(z) dz \in 2i\pi\Z$ and $\frac{1}{2i\pi}\int_{a + \partial D} z g(z)dz \in \Lambda$.\\
The Residue Theorem gives that modulo $\Lambda$ : \begin{align*}0 &= \frac{1}{2i\pi}\int_{a + \partial D} z g(z)dz\\
  &= \sum_{x\in \C/\Lambda} Res(\frac{zf'(z)}{f(z)},x) Ind(\gamma,x)\\
  &= \sum_{x\in \C/\Lambda}x v_x(f) \\
\end{align*}


\paragraph{Theorem 3.8} The map : \[\ \Psi : z + \Lambda \to (\wp(z),\wp'(z)) \]
is a bijection from nonzero points of $\mathbb{C} / \Lambda$ to the elliptic curve $E$ associated to the equation \[\ y^2 = 4x^3 - g_2x - g_3 \]
{\itshape \paragraph{} Proof :} Let $(x,y) \in E$, the value $x$ is taken twice by $\wp$ at $z_0 + \Lambda$, $-z_0 + \Lambda$ for example.
By looking at the equation we see that $(x,y)$ and $(x,-y)$ are the only two points of $E$ of the form $(x,*)$. On the other hand $(\wp(z_0+ \Lambda),\wp'(z_0+ \Lambda))$
 and $(\wp(-z_0+ \Lambda),\wp'(-z_0+ \Lambda))$ are two such points by Theorem 3.6. \\ 
 - if $ y \neq 0$, They are distinct because $\wp'(-z_0+ \Lambda) = -\wp'(z_0+ \Lambda) \neq \wp'(z_0+ \Lambda)$ 
 as $\wp'(z_0+ \Lambda) = \pm y \neq 0.$ Thus $(x,y)$ is taken exactly once by $z + \Lambda \to (\wp(z),\wp'(z))$.\\ 
 - if $y = 0$, then $\wp'(z_0 + \Lambda) = 0$. But this means that $z+ \Lambda \mapsto \wp(z+ \Lambda) - x$ 
has a double zero at $z_0+ \Lambda$ since it also has only a double pole, by Lemma 3.7, $z_0 + \Lambda$ is it's only zero. Therefore $z_0 + \Lambda = -z_0 + \Lambda$ 
and $(x,y)$ is taken exactly once by $z + \Lambda \to (\wp(z),\wp'(z))$. {\itshape This does not imply $z_0 + \Lambda = 0 + \Lambda$, for example $\omega_1/2$ is
a 2-torsion point.}

\paragraph{Definition-Proposition 3.9 (Group law)} Given an elliptic curve $E$, $\mathbb{C}/\Lambda$ its associated complex tori and two points $U,V \in E$. Let's consider the line of $\mathbb{C}^2$ going 
through $U$ and $V$ if $U \neq V$ and the tangent line to $E$ at $U$ if $U=V$. Lets define a point $O$ "at infinity" that we will consider to be
on any vertical line. To make notations coherent
 we denote $(\wp(0),\pm\wp'(0)) := O$. From now on we will identify $E$ with $E\cup\{O\}$.
Then this lines intersects $E$ at exactly one other point $W = (W_1,W_2)$.
Thus we define the group law on $E$ by : \[ \left\{ \begin{array}{rcl} U + V &=(W_1,-W_2) \\ U + O &= O + U = U \end{array}\right. \]
This makes the map $\Psi :   \left\{ \begin{array}{rcl} \mathbb{C}/\Lambda & \longrightarrow & E \\
  z + \Lambda & \longmapsto & (\wp(z),\wp'(z)) \\ \end{array} \right. $ a group isomorphism.

{\itshape \paragraph{} Proof :}
There exists two points $u,v\in\mathbb{C}/\Lambda $ such that $U=(\wp(u),\wp'(u))$ and $ V = (\wp(v),\wp'(v))$ by Theorem 3.8.
The line equation is of the form \[ ax + by + c = 0  \] with $(a,b) \neq (0,0)$. 
Lets consider the meromorphic function $f$ over $\mathbb{C}/\Lambda$ such that: \[ f(z) = a\wp(z) + b\wp'(z) + c \]

If $u,v \neq 0$ we observe that $f$ has zeros at points $u,v$ and if $u=v$ then $u$ is a zero of order 2 since the line is tangent to $E$.\\
-If $b=0$ then $f$ has a double pole at 0 and it's zeros are $u$ and $v$ by $(i)$ of the Lemma 3.7. 
and $(iii)$ gives us  $u + v + 2 \cdot 0  =0  $. In that case we let $w:= 0 $.\\
-If $b\neq 0$ then $f$ has a triple pole at 0. It has a third zero $w$ again by Lemma 3.7 and $u + v + w  = 0 $ by $(iii)$. \\
If $u$ or $v$ is 0 then define $w := - u -v$. 
We have in all cases $ u + v + w = 0$. 
And $w= 0$ or $f(w) = 0 $ so if $W := (\wp(w),\wp'(w))$ then $W$ is on the line defined by $U,V$ and 
$U+V = (\wp(w),-\wp'(w)) = (\wp(-w),\wp'(-w)) = (\wp(u + v),\wp'(u + v))$.
Therefore $\Psi(u + v) = \Psi(u) + \Psi(v)$. And $\Psi : \C/\Lambda \to E$ surjects. So $(E,+)$ is a group and $\Psi$ an isomorphism, it bijects by Theorem 3.8.

\paragraph{Theorem 3.10} Let $E$ and $E'$ be some elliptic curves associated to the lattices $\Lambda$ and $\Lambda'$ respectively,
then the following statements are equivalent :\\

  $(i)$ $E$ and $E'$ are isomorphic
  
  $(ii)$  $\mathbb{C}/\Lambda$ and $\mathbb{C}/\Lambda'$ are isomorphic.
  
  $(iii)$  $\Lambda$ and $\Lambda'$ are isomorphic. \\ 

{\itshape \paragraph{} Proof :} - $(i) \iff (ii)$ : Looking at the diagram : \\ \[
\begin{tikzcd}
\C/\Lambda \arrow[r, "\Psi"] \arrow[d,"\varphi",dashed] & E \arrow[d,"\psi",dashed] \\
\C/\Lambda' \arrow[r, "\Psi'"]         & E'         
\end{tikzcd}\]
we see that if there is such an isomorphism $\varphi$ then $\Psi' \circ \varphi \circ \Psi^{-1} :  E\to E'$ is also isomorphism. \\
Similarly if there is such an isomorphism $\psi$ then $ \Psi'^{-1} \circ \psi \circ \Psi : \mathbb{C} / \Lambda \to \mathbb{C} / \Lambda'$ is also isomorphism.\\
- $(ii) \implies (iii)$ is given by proposition 2.8.\\
- $(iii) \implies (ii)$ Let $m \in \C$ such that $m\Lambda = \Lambda'$. Then the map $\varphi : z + \Lambda \mapsto mz + \Lambda'$ is clearly a holomorphic group morphism.
We have $\Lambda = \frac{1}{m}\Lambda'$ and $\psi :  \mathbb{C} / \Lambda' \to \mathbb{C} / \Lambda$ given by $ \psi(z + \Lambda') = z/m + \Lambda $ is the inverse 
of $\varphi$. Therefore $\varphi$ is an isomorphism of complex tori.
 
{\itshape \paragraph{}This theorem reduces the problem of parametrization of elliptic curves up to isomorphism to the much simpler problem of parametrization of
 lattices up to isomorphism. In what follows we will identify those 3 categories as the same mathematical objects.
}

\section{Modular curves}%------------------------------------------------------------------------------------------------------------------------------------

{\itshape \paragraph{} In this section we are looking for a moduli spaces (i.e a space that parametrize) of elliptic curves up to isogeny. 
}



\paragraph{Definition 4.1} Let $\Gamma$ be a congruence subgroup of $SL_2(\mathbb{Z})$, we define the modular curve associated to $\Gamma$ by
 $Y(\Gamma) := \{ \Gamma\tau , \tau \in \mathcal{H} \}$ , i.e the set of orbits of the action of $\Gamma$ on $\mathcal{H}$. We also denote $Y_0(N)$ for $Y(\Gamma_0(N))$
 and $Y_1(N)$ for $Y(\Gamma_1(N))$.


\paragraph{Proposition } Each point of $Y(SL_2(\mathbb{Z}))$ correspond to a unique elliptic curve isomorphism class.

{\itshape \paragraph{}Proof : } Points $\tau, \tau' \in \mathbb{C}$ are in the same orbit iff $\tau' = \gamma \cdot \tau$ for some $\gamma \in SL_2(\mathbb{Z})$.
This is equivalent by proposition  2.4 to $\mathbb{Z} \oplus \tau\mathbb{Z}$ and $\mathbb{Z} \oplus \tau'\mathbb{Z}$ being isomorphic.



{\itshape \paragraph{} This last proposition says that $Y(SL_2(\mathbb{Z}))$ is a moduli space of elliptic curves up to isomorphism, we will now answer the question :
what classes of elliptic curves does the modular curves $Y_0(N)$ and $Y_1(N)$ represent ?  } 

\paragraph{Definition } We call a pair $(E,p)$ a pointed elliptic curve if $E$ is an elliptic curve and $p$ is a $N$-torsion point on $E$ 
(seen as an abelian quotient group $\mathbb{C}/ \Lambda$). And we say that two pointed elliptic curves $(E,p)$ and $(E',p')$ are isomorphic if there is an isomorphism
$\phi: E\to E'$ such that $\phi(p) = p'$.

\paragraph{Definition } We call a pair $(E,C)$ an enhanced elliptic curve if $E$ is an elliptic curve and $C$ is a cyclic subgroup of $E$ of order $N$. 
And we say that two enhanced elliptic curves $(E,C)$ and $(E',C')$ are isomorphic if there is an isomorphism
$\phi: E\to E'$ such that $\phi(C) = C'$.

\paragraph{Theorem }\textbf{(Moduli spaces $Y_1(N)$ and $Y_0(N)$ )} \\- Each point of $Y_1(N)$ correspond to a unique pointed elliptic curve isomorphism class.\\
 - Each point of $Y_0(N)$ correspond to a unique enhanced elliptic curve isomorphism class.

 {\itshape \paragraph{} It follows from Chow's theorem and the Rieman-Roch theorem that compact Riemann surfaces are complex algebraic curves,
  this and Theorem 1.6 gives us motivation to make $Y(\Gamma) $ a compact Rieman surface. }


\paragraph{Definition } The fundamental domain $\mathcal{D} \subset \mathcal{H}$ of the modular form $Y(\Gamma)$ is
a region that contains exactly one point from each orbit of $\Gamma$ action except some
possible duplication on the edge.

\paragraph{Proposition} The region $\mathcal{D} = \{ \tau \in \mathcal{H}\;  |\tau| \geq 1 \; and \; |Re(\tau)| \leq 1/2\}$ is a fundamental domain for $Y(SL_2(\mathbb{Z}))$.

{\itshape \paragraph{} After identification of duplicate points on the edge we see that $Y(SL_2(\mathbb{Z})) \simeq SL_2(\mathbb{Z})\backslash \mathcal{D}$ is a punctured sphere.
}

\includegraphics[scale=1.0]{sphere}

\paragraph{Proposition}  $Y(SL_2(\mathbb{Z}))$ can be made a Riemann surface and compactified into $X(1)$ by adding a point at infinity.



\paragraph{Proposition} Similarly for any congruence subgroup $\Gamma$, $Y(\Gamma)$ can be made a Riemann surface and compactified into $X(\Gamma)$
 by adding a finite number of points. We will denote $X(\Gamma_0(N))$ as $X_0(N)$ for any integer $N$. 

\paragraph{Definition} A complex elliptic curve $E$ is said to be \textbf{modular} if there exists an integer $N$ such that there is a surjection holomorphic map $\varphi$ 
from the modular curve $X_0(N)$ to $E$ as Riemann surfaces. $\varphi$ is the called a modular parametrization of $E$.

\paragraph{TODO} congruence subgroup, ex ref \cite{zhou} \cite{springer} $G_2$ Eisenstein series, convergence of $\wp$, genus...


\printbibliography %Prints bibliography

\end{document} 