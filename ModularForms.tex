\documentclass[letterpaper,10pt]{article}

% Language setting
% Replace `english' with e.g. `spanish' to change the document language
\usepackage[utf8]{inputenc}
\usepackage[T1]{fontenc}
\usepackage[english]{babel}

% Set page size and margins
% Replace `letterpaper' with `a4paper' for UK/EU standard size
\usepackage[letterpaper,top=2cm,bottom=2cm,left=3cm,right=3cm,marginparwidth=1.75cm]{geometry}

% Useful packages
\usepackage{amsmath}
\usepackage{amssymb}

%\usepackage{stmaryrd}
\usepackage{graphicx}

\usepackage{biblatex} %Imports biblatex package
%\usepackage[colorlinks=true, allcolors=blue]{hyperref}

\addbibresource{bibliography.bib} %Import the bibliography file


\title{Modular Forms, and Elliptic curves and Modular curves}
\author{Armand Perrin, Samy, Damian}

\begin{document}

\maketitle%---------------------------------------------------------------
\part{The Modularity Theorem}

\paragraph{ Theorem } (Modularity Theorem) All elliptic curves over $\mathbb{Q}$ are modular.


\paragraph{ Definition } Given a sub-field $\mathbb{K}$ of $\mathbb{C}$, an \textbf{elliptic curve} over $\mathbb{K}$ is the set of points $(x,y) \in \mathbb{K}^2$ such that 
\[\ y^2 = x^3 + ax + b \] for some $(a,b) \in \mathbb{K}^2$.

\paragraph{ Definition } A \textbf{Riemann surface} is a connected one-dimensional complex manifold.

\paragraph{Proposition} We can associate any elliptic curve $E$ with a Riemann surface : the complex tori $\mathbb{C}/\Lambda$, where $\Lambda$ 
is the latice associated to $E$.
\paragraph{Proposition} The Riemann surface $Y(\Gamma)$ can be compactified into $X(\Gamma)$ by adding a finite number of points.

\paragraph{Definition} A complex elliptic curve $E$ is said to be \textbf{Modular} if there exists an integer N such that there is a surjection morphism from the 
modular curve $X_0(N)$ to $E$ as Riemann surfaces.

\paragraph{TODO} morphisms between riemann surfaces, $Y(\Gamma),X(\Gamma),X_0(N)$
ex ref \cite{zhou}

\printbibliography %Prints bibliography

\end{document} 